\documentclass[20pt,a4paper]{extarticle}
\usepackage[utf8]{inputenc}
\usepackage[english]{babel}

\usepackage{amsmath}
\usepackage{amsfonts}
\usepackage{amssymb}


\usepackage{graphicx}
\usepackage{caption}
\usepackage{subcaption}
\usepackage{lmodern}
\usepackage{tikz}
\usetikzlibrary{calc}
\usepackage{titlesec}
\usepackage{environ}
\usepackage{xcolor}
\usepackage{fancyhdr}
\usepackage[colorlinks = true, linkcolor = black]{hyperref}
\usepackage{xparse}
\usepackage{enumitem}
\usepackage{comment}
\usepackage{wrapfig}
\usepackage[capitalise]{cleveref}

\usepackage[left=2cm,right=2cm,top=2cm,bottom=2cm]{geometry}
\usepackage{multicol}
\usepackage[indent=0pt]{parskip}

\newcommand{\spaceP}{\vspace*{0.5cm}}
\newcommand{\Span}{\mathrm{Span}\,}
\newcommand{\range}{\mathrm{range}\,}
\newcommand{\ra}{\rightarrow}
\newcommand{\curl}{\mathrm{curl} \,}
\newcommand{\hint}[1]{\scalebox{2}{$\displaystyle\int_{\scalebox{0.35}{$#1$}}$}\,}
\newcommand{\hiint}[1]{\scalebox{2}{$\displaystyle\iint_{\scalebox{0.35}{$#1$}}$}\,}
\newcommand{\hiiint}[1]{\scalebox{2}{$\displaystyle\iiint_{\scalebox{0.35}{$#1$}}$}\,}
\renewcommand{\div}{\mathrm{div}\,}

%% Redefining sections
\newcommand{\sectionformat}[1]{%
    \begin{tikzpicture}[baseline=(title.base)]
        \node[rectangle, draw] (title) {#1};
    \end{tikzpicture}
    
    \noindent\hrulefill
}

\newif\ifhNotes 

\hNotesfalse

\ifhNotes
	\newcommand{\hideNotes}[1]{%
	\phantom{#1}
	}
	\newcommand{\hideNotesU}[1]{%
	\underline{\hspace{1mm}\phantom{#1}\hspace{1mm}}
	}
\else
	\newcommand{\hideNotes}[1]{#1}
	\newcommand{\hideNotesU}[1]{\textcolor{blue}{#1}}
\fi

% default values copied from titlesec documentation page 23
% parameters of \titleformat command are explained on page 4
\titleformat%
    {\section}% <command> is the sectioning command to be redefined, i. e., \part, \chapter, \section, \subsection, \subsubsection, \paragraph or \subparagraph.
    {\normalfont\large\scshape}% <format>
    {}% <label> the number
    {0em}% <sep> length. horizontal separation between label and title body
    {\centering\sectionformat}% code preceding the title body  (title body is taken as argument)

%% Set counters for sections to none
\setcounter{secnumdepth}{0}

%% Set the footer/headers
\pagestyle{fancy}
\fancyhf{}
\renewcommand{\headrulewidth}{0pt}
\renewcommand{\footrulewidth}{2pt}
\lfoot{P.-O. Paris{\'e}}
\cfoot{MATH 244}
\rfoot{Page \thepage}

%% Defining example environment
\newcounter{example}[section]
\NewEnviron{example}%
	{%
	\noindent\refstepcounter{example}\fcolorbox{gray!40}{gray!40}{\textsc{\textcolor{red}{Example~\theexample.}}}%
	%\fcolorbox{black}{white}%
		{  %\parbox{0.95\textwidth}%
			{
			\BODY
			}%
		}%
	}

\newcounter{theorem}
% Theorem environment
\NewEnviron{theorem}%
	{%
	\noindent\refstepcounter{theorem}\fcolorbox{gray!40}{gray!40}{\textsc{\textcolor{black}{Theorem~\theexample.}}}%
	%\fcolorbox{black}{white}%
		{  %\parbox{0.95\textwidth}%
			{
			\BODY
			}%
		}%
	}

\newcounter{definition}
\NewEnviron{definition}%
	{%
	\noindent\refstepcounter{definition}\fcolorbox{gray!40}{gray!40}{\textsc{\textcolor{black}{Definition~\thedefinition.}}}%
	%\fcolorbox{black}{white}%
		{  %\parbox{0.95\textwidth}%
			{
			\BODY
			}%
		}%
	}

\NewEnviron{notes}%
	{%
	\noindent \fcolorbox{gray!40}{gray!40}{\textsc{\textcolor{blue}{Solution.}}}%
	%\fcolorbox{black}{white}%
		{  %\parbox{0.95\textwidth}%
			{
			\textcolor{blue}{%
			\BODY
			}
			}%
		}%
	}
%%% Ignorer les notes
%\excludecomment{notes}

%%%%
\begin{document}
\thispagestyle{empty}

\begin{center}
\vspace*{2.5cm}

{\Huge \textsc{Math 244}}

\vspace*{2cm}

{\LARGE \textsc{Chapter 16}} 

\vspace*{0.75cm}

\noindent\textsc{Section 16.9: Divergence Theorem}

\vspace*{0.75cm}

\tableofcontents

\vfill

\noindent \textsc{Created by: Pierre-Olivier Paris{\'e}} \\
\textsc{Fall 2023}
\end{center}

\newpage

\section{Divergence in 3D}

\begin{definition}
If $\vec{F} = \left\langle P, Q, R \right\rangle$ is a vector field in 3D, then
	\[
		\div \vec{F} = \frac{\partial P}{\partial x} + \frac{\partial Q}{\partial y} + \frac{\partial R}{\partial z} .
	\]
\end{definition}

Another way to write $\curl \vec{F}$ is as followed. Define
	\[
		\vec{\nabla} = \left\langle \frac{\partial}{\partial x} , \frac{\partial}{\partial y} , \frac{\partial}{\partial z} \right\rangle \quad \Longrightarrow \quad \div \vec{F} = \vec{\nabla} \cdot \vec{F} .
	\]

\begin{example}
Find the divergence of $\vec{F} = \left\langle xz , xyz , -y^2 \right\rangle$.
\end{example}

\begin{notes}

\end{notes}

\newpage 

\begin{theorem}
Let $\vec{F} = \left\langle P , Q, R \right\rangle$ and assume $P, Q, R$ have continuous second partial derivatives. Then 
	$$
	\div (\curl \vec{F}) = 0 .
	$$
\end{theorem}

\vspace*{0.5cm}

\begin{example}
Show that $\vec{F} (x, y, z) = \left\langle xz , xyz , -y^2 \right\rangle$ can't be written as the curl of some other vector field.
\end{example}

\begin{notes}

\end{notes}

\newpage 

\section{Divergence Theorem}

\begin{theorem}
Assume
	\begin{itemize}
	\item $S$ be a closed surface with positive orientation (outward orientation).
	\item $\vec{F} = \left\langle P , Q, R \right\rangle$ with $P, Q, R$ having continuous partial derivatives.
	\end{itemize}
Then,
	\[
		\hiint{S} \vec{F} \cdot d \vec{S} = \hiiint{E} \div \vec{F} \, dV ,
	\]
where $E$ is the solid bounded by $S$.
\end{theorem}

\vspace*{0.3cm}

\begin{example}
Let $\vec{F} (x, y, z) = \left\langle xye^z , xy^2z^3, -ye^z \right\rangle$ and $S$ is the surface of the box bounded by the coordinates planes and the planes $x = 3$, $y= 2$, and $z = 1$. Compute the flux of $\vec{F}$ across $S$.
\end{example}

\begin{notes}

\end{notes}

\newpage 

\phantom{2} 


\end{document}