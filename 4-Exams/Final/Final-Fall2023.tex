\documentclass[addpoints, 12pt]{exam}%, answers]
\usepackage[utf8]{inputenc}
\usepackage[T1]{fontenc}

\usepackage{lmodern}
\usepackage{arydshln}
\usepackage[margin=2cm]{geometry}

\usepackage{enumitem}

\usepackage{amsmath, amsthm, amsfonts, amssymb}
\usepackage{graphicx}
\usepackage{tikz}
\usetikzlibrary{arrows,calc,patterns}
\usepackage{pgfplots}
\pgfplotsset{compat=newest}
\usepackage{url}
\usepackage{multicol}
\usepackage{thmtools}
\usepackage{wrapfig}

\usepackage{caption}
\usepackage{subcaption}

\usepackage{pifont}

% MATH commands
\newcommand{\bC}{\mathbb{C}}
\newcommand{\bR}{\mathbb{R}}
\newcommand{\bN}{\mathbb{N}}
\newcommand{\bZ}{\mathbb{Z}}
\newcommand{\bT}{\mathbb{T}}
\newcommand{\bD}{\mathbb{D}}

\newcommand{\cL}{\mathcal{L}}
\newcommand{\cM}{\mathcal{M}}
\newcommand{\cP}{\mathcal{P}}
\newcommand{\cH}{\mathcal{H}}
\newcommand{\cB}{\mathcal{B}}
\newcommand{\cK}{\mathcal{K}}
\newcommand{\cJ}{\mathcal{J}}
\newcommand{\cU}{\mathcal{U}}
\newcommand{\cO}{\mathcal{O}}
\newcommand{\cA}{\mathcal{A}}
\newcommand{\cC}{\mathcal{C}}
\newcommand{\cF}{\mathcal{F}}

\newcommand{\fK}{\mathfrak{K}}
\newcommand{\fM}{\mathfrak{M}}

\newcommand{\ga}{\left\langle}
\newcommand{\da}{\right\rangle}
\newcommand{\oa}{\left\lbrace}
\newcommand{\fa}{\right\rbrace}
\newcommand{\oc}{\left[}
\newcommand{\fc}{\right]}
\newcommand{\op}{\left(}
\newcommand{\fp}{\right)}

\newcommand{\ra}{\rightarrow}
\newcommand{\Ra}{\Rightarrow}

\renewcommand{\Re}{\mathrm{Re}\,}
\renewcommand{\Im}{\mathrm{Im}\,}
\newcommand{\Arg}{\mathrm{Arg}\,}
\newcommand{\Arctan}{\mathrm{Arctan}\,}
\newcommand{\sech}{\mathrm{sech}\,}
\newcommand{\csch}{\mathrm{csch}\,}
\newcommand{\Log}{\mathrm{Log}\,}
\newcommand{\cis}{\mathrm{cis}\,}

\newcommand{\ran}{\mathrm{ran}\,}
\newcommand{\bi}{\mathbf{i}}
\newcommand{\Sp}{\mathrm{span}\,}
\newcommand{\Inv}{\mathrm{Inv}\,}
\newcommand\smallO{
  \mathchoice
    {{\scriptstyle\mathcal{O}}}% \displaystyle
    {{\scriptstyle\mathcal{O}}}% \textstyle
    {{\scriptscriptstyle\mathcal{O}}}% \scriptstyle
    {\scalebox{.7}{$\scriptscriptstyle\mathcal{O}$}}%\scriptscriptstyle
  }
\newcommand{\HOL}{\mathrm{Hol}}
\newcommand{\cl}{\mathrm{clos}}
\newcommand{\ve}{\varepsilon}

\DeclareMathOperator{\dom}{dom}

%%%%%% Définitions Theorems and al.
%\declaretheoremstyle[preheadhook = {\vskip0.2cm}, mdframed = {linewidth = 2pt, backgroundcolor = yellow}]{myThmstyle}
%\declaretheoremstyle[preheadhook = {\vskip0.2cm}, postfoothook = {\vskip0.2cm}, mdframed = {linewidth = 1.5pt, backgroundcolor=green}]{myDefstyle}
%\declaretheoremstyle[bodyfont = \normalfont , spaceabove = 0.1cm , spacebelow = 0.25cm, qed = $\blacktriangle$]{myRemstyle}

%\declaretheorem[ style = myThmstyle, name=Th\'eor\`eme]{theorem}
%\declaretheorem[style =myThmstyle, name=Proposition]{proposition}
%\declaretheorem[style = myThmstyle, name = Corollaire]{corollary}
%\declaretheorem[style = myThmstyle, name = Lemme]{lemma}
%\declaretheorem[style = myThmstyle, name = Conjecture]{conjecture}

%\declaretheorem[style = myDefstyle, name = D\'efinition]{definition}

%\declaretheorem[style = myRemstyle, name = Remarque]{remark}
%\declaretheorem[style = myRemstyle, name = Remarques]{remarks}

\newtheorem{theorem}{Théorème}
\newtheorem{corollary}{Corollaire}
\newtheorem{lemma}{Lemme}
\newtheorem{proposition}{Proposition}
\newtheorem{conjecture}{Conjecture}

\theoremstyle{definition}

\newtheorem{definition}{Définition}[section]
\newtheorem{example}{Exemple}[section]
\newtheorem{remark}{\textcolor{red}{Remarque}}[section]
\newtheorem{exer}{\textbf{Exercice}}[section]


\tikzstyle{myboxT} = [draw=black, fill=black!0,line width = 1pt,
    rectangle, rounded corners = 0pt, inner sep=8pt, inner ysep=8pt]

\begin{document}
	\noindent \hrulefill \\
	\noindent MATH-244 \hfill Created by Pierre-O. Paris{\'e}\\
	Final 120min (2h)\hfill December, Fall 2023\\\vspace*{-0.7cm}

\noindent\hrulefill

\vspace*{0.5cm}

\begin{center}
\begin{minipage}{0.6\textwidth}
\begin{Huge}
\textsc{University of Hawai'i}
\end{Huge}
\end{minipage}
\begin{minipage}{0.12\textwidth}
\includegraphics[scale=0.05]{../../../../manoaseal_transparent.png}
\end{minipage}
\end{center}
	
\vspace*{0.5cm}

\noindent\makebox[\textwidth]{\textbf{Last name:}\enspace \hrulefill}

\vspace*{0.5cm}

\noindent\makebox[\textwidth]{\textbf{First name:}\enspace\hrulefill}

\vspace*{1cm}

\begin{center}
\gradetable[h][questions]
\end{center}

\vspace*{1cm}

\noindent\textbf{Instructions:} 

\begin{itemize}
\item Write your complete name on your copy. 
\item Answer all 6 questions below.
\item Write your answers directly on the questionnaire.
\item Show ALL your work to have full credit.
\item Draw a square around your final answer.
\item Return your copy when you're done or at the end of the 2h period. 
\item No electronic devices allowed during the exam. 
\item Scientific calculator allowed only (no graphical calculators).
\item \textbf{Turn off your cellphone(s) during the exam}.
\item Lecture notes and the textbook are not allowed during the exam. 
\end{itemize}

\vspace{0.5cm}

\noindent\textbf{Your Signature:} \hrulefill

\vspace*{1.5cm}

\noindent\textsc{May the Force be with you!}\\
\textsc{Pierre Parisé}

\qformat{\rule{0.3\textwidth}{.4pt} \begin{large}{\textsc{Question}} \thequestion \end{large} \hspace*{0.2cm} \hrulefill \hspace*{0.1cm} \textbf{(\totalpoints\hspace*{0.1cm} pts)}}

\bonusqformat{\rule{0.3\textwidth}{.4pt} \begin{large}{\textsc{Bonus Question}} \end{large} \hspace*{0.2cm} \hrulefill \hspace*{0.1cm} \textbf{(\totalpoints\hspace*{0.1cm} pts)}}

\newpage % End of cover page

\begin{questions}

\newpage

\question
Here is a parametrization of a surface $S$:
  \[
    \vec{r} (u, v) = \left\langle u^3-u, v^2,u^2 \right\rangle 
  \]
for $-1 \leq u \leq 1$ and $-1 \leq v \leq 1$.
\begin{parts}
\part[5] Is the point $P = (0,0,1)$ lie on the surface $S$?
\part[5] Is the point $Q = (0,0, 1/4 )$ lie on the surface $S$?
\part[5] Find the equation of the tangent plane to the surface at $u = 1/3$, $v = 1/2$.
\part[5] Find an expression of $\vec{r}_u \times \vec{r}_v$.
\end{parts}

\newpage

\phantom{2} 

\newpage

\question 
Evaluate the following surface integrals using only the definition. Recall that
  \[
    dS = |\vec{r}_u \times \vec{r}_v| \, dA \quad \text{ and } \quad d\vec{S} = \vec{r}_u \times \vec{r}_v \, dA .
  \]
Let $S$ be the part of the plane $-2x - 3y + z = 1$ that lies above the rectangle $[0, 3] \times [0, 2]$.

\begin{parts}
\part[10] $\displaystyle \iint_S z \, dS$.
\part[10] $\displaystyle \iint_S \vec{F} \cdot d \vec{S}$, where $\vec{F} (x, y, z) = \left\langle -2x - 3y, 0, -2z \right\rangle$.
\end{parts}

\newpage 

\phantom{2} 

\newpage 

\question 
Recall that the \textit{curl} of a vector field $\vec{F} = \left\langle P , Q, R \right\rangle$ is given by
  \[
    \mathrm{curl}\, \vec{F} = \vec{\nabla} \times \vec{F} .
  \]
Using the \textit{curl}, determine wheter or not the following vector fields are conservative. \textbf{If it is conservative, find a function $f$ such that $\vec{F} = \vec{\nabla} f$}.

\begin{parts}
\part[10] $\vec{F} (x, y, z) = \left\langle z \cos y , xz \sin y , x \cos y \right\rangle$.
\part[10] $\vec{F} (x, y, z) = \left\langle 1 , \sin z , y \cos z \right\rangle$.
\end{parts}

\newpage 

\question 
Recall the identity in Stoke's Theorem:
  \[
    \iint_S \mathrm{curl} \, \vec{F} \cdot d \vec{S} = \int_C \vec{F} \cdot d \vec{r} ,
  \]
where $S$ is a surface and $C$ is the boundary (the ``edge'') of the surface.
\begin{parts}
\part[5] Let $\vec{F}$ be a generic vector field. Let $S_1$ be the surface $x^2 + y^2 + z^2 = 1$, with $z \geq 0$ and let $S_2$ be the paraboloid $z = 2 (1 - x^2 - y^2 )$. Explain why 
  $$
    \iint_{S_1} \mathrm{curl} \, \vec{F} \cdot d\vec{S} = \iint_{S_2} \mathrm{curl} \, \vec{F} \cdot d\vec{S} .
  $$
\part[15] Evaluate $\displaystyle\int_C \vec{F} \cdot d\vec{r}$ if $\vec{F} = \left\langle x + y^2 , y + z^2 , z + x^2 \right\rangle$ and $C$ is the triangle with vertices $(1, 0, 0)$, $(0, 1, 0)$, and $(0, 0, 1)$.
\end{parts}

\newpage 

\phantom{2}

\newpage

\question[10]
Recall the Divergence Theorem:
  \[
    \iint_S \vec{F} \cdot d \vec{S} = \iiint_E \mathrm{div} \, \vec{F} \, dV ,
  \]
where $S$ is a closed surface with the outward orientation and $E$ is the solid enclosed within $S$. Recall that $\mathrm{div} \vec{F} = \vec{\nabla} \cdot \vec{F}$.

Use the Divergence Theorem to compute the flux of $\vec{F} = \left\langle xye^z , xy^2 z^3 , -ye^z \right\rangle$ through the surface $S$ of the box bounded by the coordinate planes and the planes $x = 3$, $y = 2$, and $z = 1$.

\newpage 

\question[10]
Let $C$ be a generic loop that lies in the plane $x + y + z = 1$ and let $S$ be the surface enclosed by the curve in the plane $x + y + z = 1$. Show that the line integral
  \[
    \int_C z dx + 2x dy - 3y dz
  \]
is equal to zero.


%\newpage

%\rule{0.3\textwidth}{.4pt} \hspace*{0.2cm} \begin{large}{\textsc{Bonus Question}} \end{large} \hspace*{0.2cm} \hrulefill 

%Let $\vec{r} (x, y, z) = \left\langle x , y , z \right\rangle$ and $p$ be a real number. Let $\vec{F}$ be defined by
%  \[
%    \vec{F} (x, y, z) = \frac{\vec{r} (x, y, z)}{|\vec{r} (x, y, z)|^p} .
%  \]
%For which value of $p$ does $\mathrm{div} \, \vec{F} = 0$.

\end{questions}

\end{document}