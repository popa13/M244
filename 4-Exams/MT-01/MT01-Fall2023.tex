\documentclass[addpoints, 12pt]{exam}%, answers]
\usepackage[utf8]{inputenc}
\usepackage[T1]{fontenc}

\usepackage{lmodern}
\usepackage{arydshln}
\usepackage[margin=2cm]{geometry}

\usepackage{enumitem}

\usepackage{amsmath, amsthm, amsfonts, amssymb}
\usepackage{graphicx}
\usepackage{tikz}
\usetikzlibrary{arrows,calc,patterns}
\usepackage{pgfplots}
\pgfplotsset{compat=newest}
\usepackage{url}
\usepackage{multicol}
\usepackage{thmtools}
\usepackage{wrapfig}

\usepackage{caption}
\usepackage{subcaption}

\usepackage{pifont}

% MATH commands
\newcommand{\bC}{\mathbb{C}}
\newcommand{\bR}{\mathbb{R}}
\newcommand{\bN}{\mathbb{N}}
\newcommand{\bZ}{\mathbb{Z}}
\newcommand{\bT}{\mathbb{T}}
\newcommand{\bD}{\mathbb{D}}

\newcommand{\cL}{\mathcal{L}}
\newcommand{\cM}{\mathcal{M}}
\newcommand{\cP}{\mathcal{P}}
\newcommand{\cH}{\mathcal{H}}
\newcommand{\cB}{\mathcal{B}}
\newcommand{\cK}{\mathcal{K}}
\newcommand{\cJ}{\mathcal{J}}
\newcommand{\cU}{\mathcal{U}}
\newcommand{\cO}{\mathcal{O}}
\newcommand{\cA}{\mathcal{A}}
\newcommand{\cC}{\mathcal{C}}
\newcommand{\cF}{\mathcal{F}}

\newcommand{\fK}{\mathfrak{K}}
\newcommand{\fM}{\mathfrak{M}}

\newcommand{\ga}{\left\langle}
\newcommand{\da}{\right\rangle}
\newcommand{\oa}{\left\lbrace}
\newcommand{\fa}{\right\rbrace}
\newcommand{\oc}{\left[}
\newcommand{\fc}{\right]}
\newcommand{\op}{\left(}
\newcommand{\fp}{\right)}

\newcommand{\ra}{\rightarrow}
\newcommand{\Ra}{\Rightarrow}

\renewcommand{\Re}{\mathrm{Re}\,}
\renewcommand{\Im}{\mathrm{Im}\,}
\newcommand{\Arg}{\mathrm{Arg}\,}
\newcommand{\Arctan}{\mathrm{Arctan}\,}
\newcommand{\sech}{\mathrm{sech}\,}
\newcommand{\csch}{\mathrm{csch}\,}
\newcommand{\Log}{\mathrm{Log}\,}
\newcommand{\cis}{\mathrm{cis}\,}

\newcommand{\ran}{\mathrm{ran}\,}
\newcommand{\bi}{\mathbf{i}}
\newcommand{\Sp}{\mathrm{span}\,}
\newcommand{\Inv}{\mathrm{Inv}\,}
\newcommand\smallO{
  \mathchoice
    {{\scriptstyle\mathcal{O}}}% \displaystyle
    {{\scriptstyle\mathcal{O}}}% \textstyle
    {{\scriptscriptstyle\mathcal{O}}}% \scriptstyle
    {\scalebox{.7}{$\scriptscriptstyle\mathcal{O}$}}%\scriptscriptstyle
  }
\newcommand{\HOL}{\mathrm{Hol}}
\newcommand{\cl}{\mathrm{clos}}
\newcommand{\ve}{\varepsilon}

\DeclareMathOperator{\dom}{dom}

%%%%%% Définitions Theorems and al.
%\declaretheoremstyle[preheadhook = {\vskip0.2cm}, mdframed = {linewidth = 2pt, backgroundcolor = yellow}]{myThmstyle}
%\declaretheoremstyle[preheadhook = {\vskip0.2cm}, postfoothook = {\vskip0.2cm}, mdframed = {linewidth = 1.5pt, backgroundcolor=green}]{myDefstyle}
%\declaretheoremstyle[bodyfont = \normalfont , spaceabove = 0.1cm , spacebelow = 0.25cm, qed = $\blacktriangle$]{myRemstyle}

%\declaretheorem[ style = myThmstyle, name=Th\'eor\`eme]{theorem}
%\declaretheorem[style =myThmstyle, name=Proposition]{proposition}
%\declaretheorem[style = myThmstyle, name = Corollaire]{corollary}
%\declaretheorem[style = myThmstyle, name = Lemme]{lemma}
%\declaretheorem[style = myThmstyle, name = Conjecture]{conjecture}

%\declaretheorem[style = myDefstyle, name = D\'efinition]{definition}

%\declaretheorem[style = myRemstyle, name = Remarque]{remark}
%\declaretheorem[style = myRemstyle, name = Remarques]{remarks}

\newtheorem{theorem}{Théorème}
\newtheorem{corollary}{Corollaire}
\newtheorem{lemma}{Lemme}
\newtheorem{proposition}{Proposition}
\newtheorem{conjecture}{Conjecture}

\theoremstyle{definition}

\newtheorem{definition}{Définition}[section]
\newtheorem{example}{Exemple}[section]
\newtheorem{remark}{\textcolor{red}{Remarque}}[section]
\newtheorem{exer}{\textbf{Exercice}}[section]


\tikzstyle{myboxT} = [draw=black, fill=black!0,line width = 1pt,
    rectangle, rounded corners = 0pt, inner sep=8pt, inner ysep=8pt]

\begin{document}
	\noindent \hrulefill \\
	\noindent MATH-244 \hfill Created by Pierre-O. Paris{\'e}\\
	Midterm 01 \hfill 09/11/2023, Fall 2023\\\vspace*{-0.7cm}

\noindent\hrulefill

\vspace*{0.5cm}

\begin{center}
\begin{minipage}{0.6\textwidth}
\begin{Huge}
\textsc{University of Hawai'i}
\end{Huge}
\end{minipage}
\begin{minipage}{0.12\textwidth}
\includegraphics[scale=0.05]{../../../../manoaseal_transparent.png}
\end{minipage}
\end{center}
	
\vspace*{0.5cm}

\noindent\makebox[\textwidth]{\textbf{Last name:}\enspace \hrulefill}

\vspace*{0.5cm}

\noindent\makebox[\textwidth]{\textbf{First name:}\enspace\hrulefill}

\vspace*{1cm}

\begin{center}
\gradetable[h][questions]
\end{center}

\vspace*{1cm}

\noindent\textbf{Instructions:} 

\begin{itemize}
\item Make sure to write your complete name on your copy. 
\item You must answer all 5 questions below and write your answers directly on the questionnaire.
\item You have 50 minutes to complete the exam.
\item When you are done (or at the end of the 50min period), return your copy. 
\item Any electronic devices are not aloud during the exam. 
\item You can use a calculator.
\item \textbf{Turn off your cellphones during the exam}.
\item Lecture notes and the textbook are not allowed during the exam. 
\item You must show ALL your work to have full credit.
\item Draw a square around your final answer.
\end{itemize}

\vspace{0.5cm}

\noindent\textbf{Your Signature:} \hrulefill

\vspace*{1.5cm}

\noindent\textsc{May the Force be with you!}\\
\textsc{Pierre Parisé}

\qformat{\rule{0.3\textwidth}{.4pt} \begin{large}{\textsc{Question}} \thequestion \end{large} \hspace*{0.2cm} \hrulefill \hspace*{0.1cm} \textbf{(\totalpoints\hspace*{0.1cm} pts)}}

\newpage % End of cover page

\begin{questions}

\question[10]
Estimate the volume of the solid that lies below the surface $z = xy$ and above the rectangle
  \begin{align*}
  R = [0, 6] \times [0, 4] .
  \end{align*} 
Use a Riemann sum with $m = 3$ and $n = 2$, and take the sample point to be the upper right corner of each sub-rectangle.

\newpage

\question[10]
Evaluate the following iterated integral:
  \begin{align*}
  \int_0^1 \int_1^2 (x + e^{-y}) \, dx dy .
  \end{align*}

\newpage

\question[10]
Evaluate the volume of the solid that lies under the plane $4x + 6y - 2z + 15 = 0$ and above the rectangle $R = [-1, 2] \times [-1, 1]$. 

\newpage

\question[10]
Setup the integral by taking the following order: $dA = dx dy$. \textbf{Do not evaluate the integral.}

  \begin{align*}
  \iint_D y \, dA, \quad D \text{ is bounded by } y = x - 2, \, x = y^2 .
  \end{align*}

\newpage

\question[10]
Evaluate the following integral.

  \begin{align*}
  \int_0^{\sqrt{\pi}} \int_{y}^{\sqrt{\pi}} \sin (x^2) \, dx dy .
  \end{align*}


\end{questions}

\end{document}