\documentclass[12pt]{article}
\usepackage[utf8]{inputenc}

\usepackage{enumitem}
\usepackage[margin=2cm]{geometry}

\usepackage{amsmath, amsfonts, amssymb}
\usepackage{graphicx}
\usepackage{tikz}
\usepackage{pgfplots}
\usepackage{multicol}

\usepackage{comment}
\usepackage{url}
\usepackage{calc}
\usepackage{subcaption}
\usepackage{circledsteps}
\usepackage{wrapfig}
\usepackage{array}

\setlength\parindent{0pt}

\usepackage{fancyhdr}
\pagestyle{fancy}
\fancyhf{}
\renewcommand{\headrulewidth}{2pt}
\renewcommand{\footrulewidth}{0pt}
\rfoot{\thepage}
\lhead{\textsc{Math} 244}
\chead{\textsc{Homework 13}}
\rhead{Fall 2023}

\pgfplotsset{compat=1.16}

% MATH commands
\newcommand{\ga}{\left\langle}
\newcommand{\da}{\right\rangle}
\newcommand{\oa}{\left\lbrace}
\newcommand{\fa}{\right\rbrace}
\newcommand{\oc}{\left[}
\newcommand{\fc}{\right]}
\newcommand{\op}{\left(}
\newcommand{\fp}{\right)}

\newcommand{\bi}{\mathbf{i}}
\newcommand{\bj}{\mathbf{j}}
\newcommand{\bk}{\mathbf{k}}
\newcommand{\bF}{\mathbf{F}}

\newcommand{\ra}{\rightarrow}
\newcommand{\Ra}{\Rightarrow}

\newcommand{\sech}{\mathrm{sech}\,}
\newcommand{\csch}{\mathrm{csch}\,}
\newcommand{\curl}{\mathrm{curl}\,}
\newcommand{\dive}{\mathrm{div}\,}

\newcommand{\ve}{\varepsilon}
\newcommand{\spc}{\vspace*{0.5cm}}

\DeclareMathOperator{\Ran}{Ran}
\DeclareMathOperator{\Dom}{Dom}

\newcommand{\exo}[3]{\noindent\textcolor{red}{\fbox{\textbf{Section {#1}, Problem {#2}}}\hrulefill   \textbf{({#3} Pts})}\vspace*{10pt}}

\begin{document}
\thispagestyle{empty}
	\noindent \hrulefill \newline
	MATH-244 \hfill Pierre-Olivier Paris{\'e}\newline
	Homework 11 Solutions \hfill Fall 2023\newline \vspace*{-0.7cm}
	
	\noindent\hrulefill
	
	\spc
	
	\exo{16.9}{8}{20}
	\\
	We have $\dive \, \vec{F} = 3x^2 + 3y^2 + 3z^2$ and
		\[
			E = \{ (x ,y, z) \, : \, x^2 + y^2 + z^2 \leq 4 \} .
		\]
	By the Divergence Theorem,
		\begin{align*}
			\iint_S \vec{F} \, d\vec{S} = \iiint_E \dive \vec{F} \, dV &= 3 \iiint_E x^2 + y^2 + z^2 \, dV \\ 
			&= 3 \int_0^\pi \int_0^{2\pi} \int_0^2 \rho^2 \rho \sin^2 (\phi ) \, d\rho d\theta d \phi \\ 
			&= 3 \Big( \int_0^2 \rho^3 \, d\rho \Big) \Big( \int_0^{2\pi} \, d\theta \Big) \Big( \int_0^\pi \sin^2 (\phi ) \, d\phi \Big) \\ 
			&= 12 \pi^2 .
		\end{align*}

	\spc
	
	\exo{16.9}{24}{10}
	\\
	Notice that
		\[
			2x + 2y + z^2 = \left\langle 2 , 2 , z \right\rangle \cdot \left\langle x , y , z \right\rangle = \vec{F} \cdot \vec{n}
		\]
	and $\left\langle x , y ,z \right\rangle = \vec{n}$ is a normal vector to the sphere because $x^2 + y^2 + z^2 = 1$. Therefore,
		\[
			\iint_S 2x + 2y + z^2 \, dS = \iint_S \vec{F} \cdot \vec{n} dS = \iint_S \vec{F} \cdot d\vec{S}
		\]
	because $d\vec{S} = \vec{n} dS$. Using the Divergence Theorem,
		\[
			\iint_S \vec{F} \cdot d \vec{S} = \iiint_E \dive \vec{F} \, dV 
		\]
	where $E = \{ (x, y , z) \, : \, x^2 +y^2 + z^2 \leq 1 \}$. Since $\dive \vec{F} = 1$, we get
		\[
			\iint_S \vec{F} \cdot d \vec{S} = \iiint_E 1 \, dV = \mathrm{Vol} \, (E) = \frac{4 \pi}{3} .
		\]
		
	\spc 

	\exo{16.9}{27}{20}
	\\
	By the Divergence Theorem, we have
		\[
			\iint_S \mathrm{curl}\, \vec{F} \cdot d \vec{S} = \iiint_E \mathrm{div}\, ( \mathrm{curl} \, \vec{F} ) \, dV .
		\]
	Now we know that $\mathrm{div}\, (\mathrm{curl} \vec{F} ) = 0$ and therefore
		\[
			\iiint_E \mathrm{div} \, (\mathrm{curl} \, \vec{F} )\, dV = 0 .
		\]
	
\end{document}