\documentclass[12pt]{amsart}

\usepackage{xcolor}
\usepackage{hyperref}
\usepackage[margin=2.2cm, headsep=1cm, footskip=1cm]{geometry}
\usepackage{footmisc}
\usepackage{multicol}

\usepackage{fancyhdr}
\pagestyle{fancy}
\fancyhf{}
\renewcommand{\headrulewidth}{2pt}
\renewcommand{\footrulewidth}{0pt}
\rfoot{\thepage}
\lhead{\textsc{Math} 244}
\chead{\textsc{Syllabus}}
\rhead{Fall 2023}

\newcommand{\spacer}{\vspace{.2cm}}
\newcommand{\svs}{\vspace{.1cm}}

\newcommand{\red}[1]{\textcolor{red}{#1}}
\definecolor{gold}{rgb}{0.80,0.68,0.00}\newcommand{\gold}[1]{\textcolor{gold}{#1}}

\begin{document}
\thispagestyle{empty}

\begin{center}
\textsc{Math 244} \hfill {\Large\textsc{Syllabus}} \hfill \textsc{Fall 2023}
\end{center}

\noindent\hrulefill

\noindent \textbf{Lecture Section 01}: MWF 10:30--11:20am \\
Classroom: Webster 113

\spacer

\noindent\textbf{Instructor:} Pierre-Olivier Paris{\'e} (email: \texttt{parisepo@hawaii.edu})\\
Office: Physical Science Building (PSB) 302\\
Office hours: TBA

\noindent\hrulefill

\section*{Course description}
Multiple integrals, line integrals and Green's Theorem, surface integrals, Stokes' and Gauss' Theorems. \svs

\section*{Course objectives}
Upon successful completion of Math 244 the student will have an understanding of topics listed above, be able to solve routine problems in a multidimensional setting, and be able to apply properly Green's, Stokes, and Gauss' Theorems to compute line integrals, or surface integrals.

\noindent{\bf Prerequisites:}
Math 243 or consent.

\section*{Required course material}

\noindent{\bf Book:} \emph{Calculus}, 8th Edition, by Stewart (ISBN-13: 9781285740621). 

\noindent{\bf Course website:} \url{https://mathopo.ca/courses-website/MATH-244/MATH-244.html}. All the information about the course (like the schedule, important dates, lecture notes) is posted on the course website. All homework assignments and announcements are posted on the Laulima website.

\section*{Important Dates}

	\begin{itemize}
	\item Midterms:
		\begin{itemize}
		\item Midterm 1 on September 11th.
		\item Midterm 2 on October 16th.
		\item Midterm 3 on November 13th.
		\end{itemize}
	\item Final:
		\begin{itemize}
		\item December, 11th, 9:45--11:45am (classroom unknown yet).
		\end{itemize}
	\item Non-instructional day(s):
		\begin{itemize}
		\item Labor Day, September 4th.
		\item Veteran's Day, November 10th.
		\item The day after Thanksgiving Day, November 24th.
		\end{itemize}
	\end{itemize}

\section*{Grading components}
Your final grade will be calculated based on a weighted average of the following components.

\begin{enumerate}
\item{\bf Summaries (10\%):} There will be two summaries of the material from the Chapter 12. The first summary is due on August, 25th, 6pm. The second summary is due on the week after the week we finish covering the material of chapter 15. The detailed instructions are available on Laulima, under the Assignments tab.
%\item{\bf Quizzes (10\%):} There will be quizzes every week, at the beginning of each Monday lecture, starting on September 4th. Your best 12 quiz scores will be considered to compute your average.
\item{\bf Homework (40\%):} There will be homework each week assigned on each Mondays (starting on August, 21st) and due for the next Monday, before 6pm. The problems will be posted on Laulima under the Assignments tab. Your best 10 homework scores will be considered to compute your average.
\item {\bf Midterm exams (30\%):} There will be 3, closed notes, closed book, non-cummulative midterms of 50min each. Your best 2 exam scores will be considered to compute the average.
\item {\bf Final exam (20\%):} There will be a final comprehensive exam as scheduled by the university on December, 11th, 9:45--11:45am. This exam is in-person.
\end{enumerate}

\section*{Lectures}
If you miss a lecture, you are responsible for any assignments and/or announcements made. Unavoidable absences should be explained to the instructor. Office hours will not be utilized to re-teach material presented in a class missed. Also the lectures won't be recorded, but the notes taken in class will be posted on the course website.

\section*{Missed assignment policies}

\noindent{\bf Policies for exams:} Attendance on the exams is compulsory; otherwise, a grade of zero will be recorded. Any student who has an excused, documented conflict with a test time must inform their instructor \textbf{within two week prior to the exam targeted} when possible.  Late requests will be denied.

For those students with an excused absence for an exam, there will be a make-up exam. \textbf{For any reason, the final exam cannot be taken ahead of the scheduled time.}

Conflicts arising from work or social obligations, or from personal travel plans \textbf{do not} qualify as excused absences. By registering for this course, you are agreeing to take all exams at the scheduled times.

\noindent{\bf Policies for homework:} Late homework won't be accepted and will result of a mark of zero(0). If the guideline for a homework is not followed, at most 5\% will be removed from this homework total points. The guideline is available in the Homework webpage of the course website.
\svs

\noindent{\bf Policies for Summaries:}  Late summaries are accepted, but 20\% of the total points will be deducted for each late day.

\noindent{\bf Academic integrity:}
All students are expected to abide by the university's Conduct Code. Academic sanctions for dishonesty may include receiving an F in the assignment or receiving an F in the class. There may be additional administrative sanctions.

\section*{Classroom policies}
Tablets can be used for note-taking only. During lectures, to respect the lecturer and your classmates, refrain from using electronic items, cell phones, music players, tablets, and laptops for any other purposes than note-taking.

Please arrive, be seated and ready to start each class on time. If you have a valid reason to leave early, please advise me before the class and try to sit near the exit to minimize disruption.

\section*{Sources of help}

{\bf KOKUA:} I am happy to work with you and the KOKUA Program (Office for Students with Disabilities), if you need course accommodations due to a disability. KOKUA can be reached at (808) 956-7511 or (808) 956-7612 (voice/text) in room 013 of the Queen Lili`uokalani Center for Student Services. All course modifications must be arranged through KOKUA. You are encouraged to start this process as early as possible.

\section*{Concerns}
If at any time during the semester you have any questions or concerns about the class, please contact me during regularly scheduled office hours or via email to make an appointment. You may also contact the following people:
\spacer

\noindent {\bf Director of Undergraduate Studies}\\
Mirjana Jovovic \\
Email: \texttt{undergrad-dir@math.hawaii.edu}

\svs
\noindent {\bf Associate Chair}\\
Bj{\o}rn Kjos-Hanssen \\
Email: \texttt{assoc-chair@math.hawaii.edu}

\end{document}

